\documentclass[a4paper]{article}
\begin{document}
\title{Code Organisation for Research Software}
\author{Oxford Research Software Engineering Group}
\maketitle

\section{Introduction}
You may well have encountered problems associated with developing code with no structure in mind. When all of your program is in a single file it becomes hard to find the part you need to change to fix a bug. When you do make a change, it's difficult to understand all of the ramifications of that change, and whether you have changed everything that needs updating.

In this course, you'll learn some basics of code organisation. Firstly you'll learn about some concepts that software engineers use when thinking about how to find parts of a big chunk of code to break out into separate routines, procedures or functions. Then you'll be given practical steps to apply those concepts and organise your code.

With code organised into distinct functions that serve different purposes, you will then learn to separate these functions out into their own files, and import them where you need to use them. This lets you reuse the same functions across different projects, increasing the impact of your software and your efficiency as a programmer.

Finally, once you've learned how to import code from other files, you'll learn how to import code from libraries written by other authors, further increasing your efficiency by allowing you to stand on the shoulders of other programming giants.

All of the examples in this handout use the Python programming language.

\section{Designing Well-Organised Software}



\end{document}
